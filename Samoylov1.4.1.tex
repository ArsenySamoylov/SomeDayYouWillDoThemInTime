\documentclass[a5paper, 11pt]{article}

\usepackage[utf8]{inputenc}

\usepackage[utf8]{inputenc}
\usepackage[T2A]{fontenc}
\usepackage[english, russian]{babel}
\usepackage{graphicx}
\graphicspath{{./images/}}

\usepackage{multirow}
\usepackage{float}
\usepackage{amsthm}
\usepackage{amsmath}
\usepackage{amssymb}
\usepackage{tikz}
\usepackage{textcomp}

\usepackage{wrapfig}

\usepackage[english, russian]{babel}

\usepackage[left=3cm,right=3cm,
top=1cm,bottom=1cm,bindingoffset=0cm]{geometry}

\setlength{\topmargin}{-0.5in}
\setlength{\textheight}{9.1in}
\setlength{\oddsidemargin}{-0.4in}
\setlength{\evensidemargin}{-0.4in}
\setlength{\textwidth}{7in}
\setlength{\parindent}{0ex}
\setlength{\parskip}{1ex}

\title{\textbf{Отчёт о выполнении лабораторной работы 1.4.1} \\
	Физический маятник}
\author{Cамойлов Арсений Сергеевич, Б01-206}
\date{\today}


\begin{document}
	
	\maketitle
	
	\section{Введение}
    \textbf{Цель работы:}
    
    
	на примере измерения периода свободных колебаний физического
	маятника познакомиться с систематическими и случайными погрешностями, прямыми и косвенными измерениями;
	 
	\\
	
	\\
	\\
	
	\textbf{Оборудование:} 
	
	
	металлический стержень с опорной призмой; закреплённая на стене консоль; подставка с острой гранью для определения цента масс маятника; секундомер; электронные весы; 
	\newpage
	\section{Теоретические сведения}
	
		
     Движение маятника описывается уравнением
	\begin{equation}
		I\ddot{\varphi}=M,
		\label{osnova}
	\end{equation}
	\noindent где $I$ -- момент инерции маятника, $\varphi$ -- угол отклонения маятника от положения равновесия, $t$ - время, М - момент сил, действующих на маятник.
	
	В данной работе в качестве физического маятника используется однородный стальной стержень длиной $l$. 
	Расчет момента инерции для стержня при вращении вокруг препендикулярной стержню оси. 
	
На стержне закрепляется опорная призма, острое ребро которой является осью качания маятника.
Момент импульса можно расчитать по формуле:
	\begin{equation}
		I=\frac{ml^2}{12}+ma^2,
	\end{equation}
	
	\noindent где $ m $ -- масса маятника. $ a $ -- расстояние от точки вращения до центра масс,

	
	\noindent  Моментом силы трения можно пренебречь. 
	
	Тогда период колебаний равен
	\begin{equation}\label{period1}
		T=\frac{2\pi}{\omega}=2\pi\sqrt{\frac{a^2+\frac{l^2}{12}}{ag}}
	\end{equation}
	
	Таким образом, период малых колебаний не зависит ни от начальной фазы, ни от амплитуды колебаний. 
	
	Период колебаний математического маятника определяется формулой
	\begin{equation}
		T'=2\pi\sqrt{\frac{l'}{g}},
	\end{equation}
	где $ l' $ -- длина математического маятника.
	\par Величину
	\begin{equation}\label{prived}
		l_\text{пр}=a+\frac{l^2}{12a}
	\end{equation}
	называют приведённой длиной математического маятника. 
	
	\par 
	Формула (3) получена в предположении, что подвес маятника является
	материальной точкой. На самом же деле маятник подвешивается с помощью треугольной призмы конечного размера, поэтому использование (3)
	может привести к систематической погрешности результата. Для более
	точных расчётов нужно принять во внимание наличие призмы:
	\begin{equation}
		T=\frac{2\pi}{\omega}=2\pi\sqrt{\frac{ma^2+\frac{ml^2}{12} + m_\text{пр}\frac{b^2}{a}}{(m + m_\text{пр})a_{c}g}}
	\end{equation}
	\noindent где $m_{пр}$ -- масса маятника, ${a_{c}}$ - расстояние от призмы до центра масс системы. Слагаемым $m_{пр}\frac{b^2}{a}$ можно пренебречь.

	Отсюда, получаем формулу с нужной нам поправкой:
	\begin{equation}\label{period2}
		T=2\pi\sqrt{\frac{a^2+\frac{l^2}{12}}{(1 + \frac{m_\text{пр}}{m})a_{c}g}}
	\end{equation}
\section {Экспериментальная установка}
Тонкий стальной стержень подвешивается на прикреплённой стене консоли с помощью небольшой призмы. Диаметр стержня много меньше его длины. Небольшая призма крепится на стержне винтом и острым основанием опирается на поверхность закреплённой на стене консоли. Острие ребра призмы образует ось качания маятника.

 Призму можно перемещать вдоль стержня, изменяя
расстояние от центра масс до точки подвеса. Период колебаний измеряем непосредственно с помощью секундомера для разных положений призмы на стержне.

\newpage
\section{Ход работы:}
\subsection{Оценка погрешностей измерительных приборов и $g$}
\textbf{Секундомер:} $ \sigma_c = 0,01 \text{ с}$\\
\textbf{Линейка:} $ \sigma_\text{лин} = 0,05 \text{ см}$\\\\

Погрешность конечного результата (величины g) определяется точностью измерения длин и периода колебаний. Длины измеряются металлической линейкой. Абсолютное значение погрешности металлической линейки $\sigma_{\text{лин}}$ = 0,5 мм. Тогда относительная погрешность измерения длин составит по порядку величины $\varepsilon_\text{max}\approx0,1\%$.

\subsection{Длина стержня и массы груза, штанги и призмы}

Длина стержня $l = \left(  100,00 \pm 0,05\right)$ см, 
Масса стержня $m =\left(  1022,4 \pm 0,1\right)$ г, 
Масса призмы $m_\text{пр}=\left( 74,9 \pm 0,1\right)$ г. 

\subsection{Центр масс стержня и конструкции}
Центр масс стержня расположен на расстоянии $l = (50, 00 \pm 0,05) \text{ см}$ от одного из его концов.  
Сбалансировав маятник \emph{с призмой} на острие вспомогательной установки, будем измерять положение центра масс конструкции для каждого положения призмы. Определение точного положения центра масс усложняется тем, что достичь точного равновесия конструкции на установке почти невозможно, поэтому погрешность измерения будет составлять около $0,1\text{ см}$.

\subsection{Предварительный опыт}
Установим маятник на консоли и отклоним его на малый угол. Измерим время $n = 14$ полных колебаний и вычислим период колебаний $T = t/n$. Результаты 10 измерений приведем в таблице 1.
\begin{table}[!ht]
	\centering
	\begin{tabular}{|l|l|l|l|l|l|l|l|l|l|l|}
		\hline
		№ измерения  & 1 & 2 & 3 & 4 & 5 & 6 & 7 & 8 & 9 & 10  \\ \hline
		n, кол-во колебаний  & 14 & 14 & 14 & 14 & 14 & 14 & 14 & 14 & 14 & 14  \\ \hline
		t, с  & 22,56 & 22,56 & 22,62 & 22,63 & 22,63 & 22,66 & 22,59 & 22,63 & 22,60 & 22,57  \\ \hline
	\end{tabular}
	\caption{Результаты измерений периода колебаний}
\end{table}

\par Систематическая погрешность измерения времени составляет
\begin{equation}
	\sigma_\text{сист} = 0,010\text{ c}.
\end{equation}
\par Случайная погрешность измерения времени составляет
\begin{equation}
	\sigma_\text{случ} = \sqrt{\frac{1}{N} \sum_{i=1}^{N} \left( t_i - \langle t \rangle \right)^2 } \approx 0,034 \text{ с}
\end{equation}
\par Полная погрешность измерения времени составляет
\begin{equation}
	\sigma_\text{t} = \sqrt{\sigma_\text{случ}^2+\sigma_\text{сист}^2} = 0,035\text{ c}.
\end{equation}

\par Следовательно, с учетом погрешности период колебаний равен $T = 1,6146 \pm 0,0025$ с.



\subsection{Измерение периода колебаний для различных значений $\text{a}$}

С помощью секундомера будем измерять время, необходимое для n колебаний, для разных положений призмы относительно центра масс стержня. Для увеличения точности будем измерять $t$ для каждого положения призмы 3 раза с последующим усреднением. Результаты приведены в таблице 2.

\begin{table}[H]
	\centering
	\begin{tabular}{|l|l|l|l|l|l|l|l|l|l|l|l|l|}
		\hline
		k & n & $t_1,$ с & $t_2,$ с & $t_3,$ с & $t_\text{ср},$ с & T, с & a, см & $a_c,$ см & $T^2 \cdot a,$ м$\cdot c^2$ & \begin{tabular}[c]{@{}l@{}}$T^2 \cdot (1 + \frac{m_{\text{пр}}}{m}) \cdot$ \\ $\cdot a_c,$ м$\cdot c^2$\end{tabular} & $a^2, \text{м}^2$ \\ \hline
		1 & 15 & 23,66 & 23,72 & 23,68 & 23,686 & 1,5791 & 41,6 & 38,6 & 1,0373 & 1,0330 & 0,1731  \\ \hline
		2 & 15 & 23,53 & 23,50 & 23,54 & 23,523 & 1,5682 & 39,7 & 36,9 & 0,9764 & 0,9740 & 0,1577  \\ \hline
		3 & 15 & 23,34 & 23,34 & 23,35 & 23,343 & 1,5562 & 37,7 & 35,0 & 0,9130 & 0,9097 & 0,1421  \\ \hline
		4 & 15 & 23,18 & 23,19 & 23,18 & 23,183 & 1,5456 & 35,5 & 32,9 & 0,8480 & 0,8435 & 0,1260  \\ \hline
		5 & 15 & 23,12 & 23,08 & 23,10 & 23,100 & 1,5400 & 33,6 & 31,2 & 0,7969 & 0,7941 & 0,1129  \\ \hline
		6 & 15 & 22,96 & 22,97 & 22,94 & 22,956 & 1,5304 & 30,5 & 28,3 & 0,7144 & 0,7114 & 0,0930  \\ \hline
		7 & 15 & 22,91 & 22,90 & 22,90 & 22,903 & 1,5269 & 28,1 & 26,2 & 0,6551 & 0,6543 & 0,0790  \\ \hline
		8 & 15 & 23,12 & 23,12 & 23,06 & 23,100 & 1,5400 & 24,6 & 22,8 & 0,5834 & 0,5803 & 0,0605  \\ \hline
		9 & 15 & 23,28 & 23,28 & 23,25 & 23,270 & 1,5513 & 22,6 & 21,0 & 0,5439 & 0,5424 & 0,0511  \\ \hline
		10 & 15 & 23,75 & 23,78 & 23,87 & 23,800 & 1,5867 & 19,6 & 18,2 & 0,4934 & 0,4918 & 0,0384  \\ \hline
		11 & 15 & 24,69 & 24,69 & 24,68 & 24,686 & 1,6458 & 16,6 & 15,4 & 0,4496 & 0,4477 & 0,0276  \\ \hline
		12 & 15 & 26,03 & 26,07 & 26,09 & 26,063 & 1,7376 & 13,7 & 12,8 & 0,4136 & 0,4148 & 0,0188  \\ \hline
	\end{tabular}
	\caption{Измерения периода колебаний физического маятника}
\end{table}
k - Номер опыта\\
n - Количество колебаний\\
$t_1$ - Измерение №1\\
$t_2$ - Измерение №2\\
$t_3$ - Измерение №3\\
$t_\text{ср}$ - Среднее время на n колебаний\\
T - период колебаний\\
a -  Расстояние от точки подвеса до центра масс стержня без учета призмы\\
$a_c$ -  Расстояние от точки подвеса до центра масс стержня с учетом призмы        

\section{Обработка результатов измерений}
\subsection{Анализ зависимостей}
Без учета влияния призмы период колебаний равен:

\begin{equation}
	T=2\pi\sqrt{\frac{a^2+\frac{l^2}{12}}{ag}}
\end{equation}
Преобразуем:
\begin{equation}
	T^2 \cdot a=\frac{4\pi^2}{g}(a^2+\frac{l^2}{12})=\frac{4\pi^2}{g}\cdot a^2+\frac{4\pi^{2}l^2}{12g}
\end{equation}

Заметим, что формула (12) представляет собой линейную зависимость $T^2 \cdot a$ от $a^2$.


С учетом влияния призмы:
\begin{equation}\label{period2}
	T=2\pi\sqrt{\frac{a^2+\frac{l^2}{12}}{(1 + \frac{m_\text{пр}}{m})a_{c}g}}
\end{equation}
Преобразуем:
\begin{equation}
	T^2 (1 + \frac{m_\text{пр}}{m}) a_{c}=\frac{4\pi^2}{g}(a^2+\frac{l^2}{12})=\frac{4\pi^2}{g}\cdot a^2+\frac{4\pi^{2}l^2}{12g}
\end{equation}
Заметим, что формула (14) представляет собой линейную зависимость $T^2 (1 + \frac{m_\text{пр}}{m}) a_{c}$ от $a^2$.

\subsection{Графики}

Используя таблицу 2, посторим график зависимости периода колебаний $T$ от расстояния
подвеса $a$. Можно заметить, что зависимость имеет минимум. В соответствии с графиком можем видеть, что значение минимума для периода составляет $T_{min} \approx 1,52$ c.

\begin{figure}[H]
	
	\centering
	\includegraphics[width=1\linewidth]{laba1..png}
	\caption{график зависимости $T$ от $a$}
	
\end{figure}

\newpage
Построbм график, откладывая по оси абсцисс величину $u = T^2 \cdot a$, а по оси ординат величину $v = a^2$. Можем видеть, что экспериментальные точки графика хорошо ложатся на прямую линию.
\begin{figure}[H]
	
	\centering
	\includegraphics[width=1\linewidth]{laba2.png}
	\caption{график зависимости $T^2 \cdot a$ от $a^2$}
	
\end{figure}

\newpage
Теперь учтем влияние призмы. Для этого постром график, откладывая по оси абсцисс величину $T^2 (1 + \frac{m_\text{пр}}{m}) a_{c}$, а по оси ординат величину $v = a^2$. Экспериментальные точки снова хорошо ложатся на прямую.

\begin{figure}[H]
	
	\centering
	\includegraphics[width=1\linewidth]{laba3.png}
	\caption{график зависимости $T^2 (1 + \frac{m_\text{пр}}{m}) a_{c}$ от $a^2$}
	
\end{figure}

\newpage
\subsection{Метод наименьших квадратов}

Методом наименьших квадратов определим параметры ($k, b$) наилучшей прямой $u = kv + b$ и их погрешности ($\sigma_k$ и $\sigma_b$).

\[k = \frac{\langle uv\rangle - \langle u\rangle \langle v\rangle}{\langle v^2 \rangle - \langle v \rangle^2} \]

\[b = \langle u \rangle - k\langle v \rangle \]

Погрешности этих коэффициентов соответственно равны:
\[\sigma_k \approx \frac{1}{\sqrt{N}}\sqrt{\frac{\langle u^2 \rangle - \langle u \rangle^2} {\langle v^2 \rangle - \langle v \rangle^2} - k^2} \]

\[\sigma_b = \sigma_k \sqrt{\langle v^2 \rangle - \langle v \rangle^2} \]

$k_1 \approx 4,048\text{ }\frac{\text{с}^2}{\text{м}}, b_1 \approx 0,338 \text{ }\text{м}\cdot\text{с}^2$  

$\sigma_{k_1} \approx 0,0063\cdot \text{ }\frac{\text{с}^2}{\text{м}},  \sigma_{b_1} \approx 6,449 \cdot 10^{-4} \text{ }\text{м}\cdot\text{с}^2$

$k_2 \approx 4,027\text{ }\frac{\text{с}^2}{\text{м}}, b_2 \approx 0,337 \text{ }\text{м}\cdot\text{с}^2 $ 

$\sigma_{k_2}  \approx 0,0077 \text{ }\frac{\text{с}^2}{\text{м}},  \sigma_{b_2} \approx 7,91 \cdot 10^{-4} \text{ }\text{м}\cdot\text{с}^2$

В соответствии с формулами (17) и (19):
\begin{equation}
	g = \frac{4\pi^2}{k}
\end{equation}

По коэффициентам k рассчитаем величину ускорения свободного падения:
 
$g_{1} = 9,753 \frac{\text{м}}{c^2},$ 
$g_{2} = 9,803 \frac{\text{м}}{c^2}$

Погрешность для $g$:

$\sigma_{g_1} = \varepsilon_{k_1} \cdot g = \frac{\sigma_k}{k}g = 0,015 \frac{\text{м}}{c^2}$

$\sigma_{g_2} = \varepsilon_{k_2} \cdot g = \frac{\sigma_k}{k}g = 0,019 \frac{\text{м}}{c^2}$

$\varepsilon_{g_1} = \frac{\sigma_g}{g_1} \approx 0,15\%$

$\varepsilon_{g_2} = \frac{\sigma_g}{g_2} \approx 0,19\%$

Таким образом, метод наименьших квадратов при обработке данных обеспечил хорошую точность для значения $g$ .


\section{Вывод}
Проделанный опыт подтверждает теорию для периода колебаний физического маятника и теорию о приведенной длине физического маятника. 

В ходе работы мы получили следующие величины:
\begin{itemize}
	\item $ g_1 = \left( 9,753\pm 0,015\right) \frac{\text{м}}{\text{с}^2} $, $ \varepsilon_{g_1} = 0,15\% $
	
	\item $ g_2 = \left( 9,803\pm 0,019\right) \frac{\text{м}}{\text{с}^2} $, $ \varepsilon_{g_2} = 0,19\% $
\end{itemize}

Значение $g_2$, полученное с учетом влияния призмы и являющееся наиболее точным в рамках проведенного экперимента, с хорошей точностью (в пределах погрешности) совпадает с табличным значением $g = 9,8067 \frac{\text{м}}{c^2}$.


\end{document}
